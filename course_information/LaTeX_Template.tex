%%%%%%%%%%%%%%%%%%%%%%%%%%%%%%%%%%%%%%%%%%%%%%%%%%%%%%%%%%%%%
% LaTeX Template
% -----------------------------------------------------------
% This template introduces the basics of LaTeX for writing 
% reports, homework, and projects
%
% NOTE: Any line that begins with "%" is a COMMENT.
% Comments are ignored by LaTeX and are only here to help you.
%%%%%%%%%%%%%%%%%%%%%%%%%%%%%%%%%%%%%%%%%%%%%%%%%%%%%%%%%%%%%

% -----------------------------------------------------------
% 1. DOCUMENT CLASS
% -----------------------------------------------------------
% The document class sets the overall formatting (font, spacing, etc.).
% "article" is good for reports, papers, and homework.
\documentclass[12pt]{article} 

% Options like 12pt change the base font size. Alternatives: 10pt, 11pt.

% -----------------------------------------------------------
% 2. PACKAGES
% -----------------------------------------------------------
% Packages extend LaTeX’s functionality, similar to Python. You load them with \usepackage.

% Better math formatting:
\usepackage{amsmath, amssymb}

% To include images (figures/plots):
\usepackage{graphicx}

% To handle color in text, tables, etc.:
\usepackage{xcolor}

% To improve table formatting:
\usepackage{booktabs}

% For including code snippets in a nice style:
\usepackage{listings}

% Control page layout (margins):
\usepackage[margin=1in]{geometry}

% -----------------------------------------------------------
% 3. TITLE INFO
% This makes your title, but you will still need to print it (see below)
% -----------------------------------------------------------
\title{Sample Data Science Report in LaTeX}
\author{Your Name Here}
\date{\today} % \today automatically prints the current date

% -----------------------------------------------------------
% 4. BEGIN DOCUMENT
% -----------------------------------------------------------
\begin{document}

% Print the title/author/date info that is written above:
\maketitle

% -----------------------------------------------------------
% 5. INTRODUCTION SECTION
% -----------------------------------------------------------
\section{Introduction}

This is a sample \LaTeX{} document for a data science class. 
We will write text, add math, tables, figures, and even code.
If you write your text on two lines in the editor
LaTeX does not recognize the enter.
You must hit enter twice

for your text to go to a second line.

By default, LaTeX indents a second paragraph. Use 

\noindent if you don't want your second line to be indented




Beyond this, the number of lines separating text does nothing

\vspace{1em} \noindent This adds a vertical space with no indent


% Notice how we just type plain text. LaTeX will handle line breaks
% and spacing automatically. You don’t press enter at the end of
% each line unless you want to start a new paragraph.

% -----------------------------------------------------------
% 6. MATH EXAMPLES
% -----------------------------------------------------------
\section{Math Examples}

You can write math inline like this: $y = mx + b$.

Or you can display equations centered on their own line:

\begin{equation}
    \hat{y} = \beta_0 + \beta_1 x_1 + \beta_2 x_2 + \epsilon
\end{equation}

% "equation" automatically numbers your equation.

Use this if you don't want the number

\begin{equation*}
    \hat{y} = \beta_0 + \beta_1 x_1 + \beta_2 x_2 + \epsilon
\end{equation*}

% let's continue on a new page
\newpage


% -----------------------------------------------------------
% 7. TABLE EXAMPLE
% -----------------------------------------------------------
\section{Tables}

Here is a simple table with nice formatting:

\begin{table}[h] % [h] means "place here"
    \centering
    \begin{tabular}{lrr} % l=left, r=right, c=center alignment
        \toprule
        Variable & Mean & Std. Dev. \\
        \midrule
        Age      & 34.5 & 5.2 \\
        Income   & 55000 & 12000 \\
        \bottomrule
        Score    & 78.9 & 10.3 \\
    \end{tabular}
    \caption{Summary statistics for example dataset.}
    \label{tab:summary}
\end{table}

% -----------------------------------------------------------
% 8. FIGURE (IMAGE) EXAMPLE
% -----------------------------------------------------------
\section{Figures}

To include an image (e.g., plot, diagram, or photo):

\begin{figure}[h]
    \centering
    \includegraphics[width=0.6\textwidth]{WVU.png}
    % width=0.6 will adjust the size of the figure
    \caption{An example figure.}
\end{figure}

% Note: The image file must be in the same folder as your .tex file,
% or you must give the path.


\newpage

% -----------------------------------------------------------
% 9. CODE SNIPPET EXAMPLE
% -----------------------------------------------------------
\section{Code Example}

We can include code with the "listings" package from line 38:

\begin{lstlisting}[language=Python, caption=Example Python code]
import pandas as pd

# Load CSV file
df = pd.read_csv("data.csv")

# Show first rows
print(df.head())
\end{lstlisting}

% You can change "language=Python" to R, SQL, etc.

% -----------------------------------------------------------
% 10. END DOCUMENT
% -----------------------------------------------------------

% You must start and end everything in LaTeX, including the document itself
\end{document}

